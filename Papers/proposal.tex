\documentclass{acmsiggraph}

\usepackage[scaled=.92]{helvet}
\usepackage{times}
\usepackage{graphicx}
\usepackage{parskip}
\usepackage[labelfont=bf,textfont=it]{caption}
\onlineid{0}

\title{Cinematic Particle Systems with OpenCL}

\author{Tim Horton\thanks{e-mail: hortot2@rpi.edu}\\Rensselaer Polytechnic Institute}

\begin{document}

\maketitle

\section{Proposal}

As my final project for Advanced Computer Graphics, I'm going to design, implement, and analyze a set of tools for creating, simulating, and rendering offline particle systems. The simulation and rendering tools will utilize OpenCL in order to take advantage of the incredibly parallelizable nature of this problem domain and the very powerful graphics hardware that ships with nearly every computer today.

\section{Software and Hardware}

All software will be built and tested on Mac OS 10.6 (Snow Leopard) and will likely not run on other platforms due to dependence on Apple-specific libraries. However, it may be possible to port the simulation and rendering components to other POSIX platforms which have OpenCL drivers with some effort.

All benchmarks and tests will be run on a machine with an Intel Core 2 Duo E7200 CPU running at 2.53 GHz, 8GB of DDR2-1066 RAM, an ATI Radeon HD 4890 graphics card with 1GB of video memory, and an array of disks providing approximately 150MB/s read and 80MB/s write throughput. All of these specifications contribute significantly to the measured performance of the code, so a different configuration could generate wildly different statistics.

\section{Examples}

During initial development, I've been using an n-body particle simulation as my primary test case. While it's extremely simple to generate and compute, it's incredibly computationally intensive (each particle acts on each other particle), so it's a prime test of the speedup gained by using OpenCL. I generally test the n-body simulation --- which is $\textrm{O}(n^2)$ --- with 4,096 particles, bumping the count up to 32,768 when I want to strain performance; when simulating $\textrm{O}(n)$ forces, one million particles is relatively painless.

Later on, I plan to use various yet-to-be-designed arrangements of other forces and emitters, double-checking my work against Blender's particle simulation engine.

\section{Components}

\subsection{Interpolator}

Interpolator is the primary GUI component of this project. Written in Objective-C against Cocoa, it provides a b-spline editor and a simple way to map curves to arbitrary properties on emitters or forces.

\subsection{Previewer}

Duis autem vel eum iriure dolor in hendrerit in vulputate velit esse
molestie consequat, vel illum dolore eu feugiat nulla facilisis at
vero eros et accumsan et iusto odio dignissim qui blandit praesent
luptatum zzril delenit augue duis dolore te feugait nulla
facilisi. Lorem ipsum dolor sit amet, consectetuer adipiscing elit,
sed diam nonummy nibh euismod tincidunt ut laoreet dolore magna
aliquam erat volutpat.

\subsection{Renderer}

Duis autem vel eum iriure dolor in hendrerit in vulputate velit esse
molestie consequat, vel illum dolore eu feugiat nulla facilisis at
vero eros et accumsan et iusto odio dignissim qui blandit praesent
luptatum zzril delenit augue duis dolore te feugait nulla
facilisi. Lorem ipsum dolor sit amet, consectetuer adipiscing elit,
sed diam nonummy nibh euismod tincidunt ut laoreet dolore magna
aliquam erat volutpat.

\subsection{Simulator}

Duis autem vel eum iriure dolor in hendrerit in vulputate velit esse
molestie consequat, vel illum dolore eu feugiat nulla facilisis at
vero eros et accumsan et iusto odio dignissim qui blandit praesent
luptatum zzril delenit augue duis dolore te feugait nulla
facilisi. Lorem ipsum dolor sit amet, consectetuer adipiscing elit,
sed diam nonummy nibh euismod tincidunt ut laoreet dolore magna
aliquam erat volutpat.

\section{Timeline}

Lorem ipsum dolor sit amet, consetetur sadipscing elitr, sed diam
nonumy eirmod tempor invidunt ut labore et dolore magna aliquyam erat,
sed diam voluptua. At vero eos et accusam et justo duo dolores et ea
rebum. Stet clita kasd gubergren, no sea takimata sanctus est Lorem
ipsum dolor sit amet. Lorem ipsum dolor sit amet, consetetur
sadipscing elitr, sed diam nonumy eirmod tempor invidunt ut labore et
dolore magna aliquyam erat, sed diam voluptua. At vero eos et accusam
et justo duo dolores et ea rebum. Stet clita kasd gubergren, no sea
takimata sanctus est Lorem ipsum dolor sit amet. Lorem ipsum dolor sit
amet, consetetur sadipscing elitr, sed diam nonumy eirmod tempor
invidunt ut labore et dolore magna aliquyam erat, sed diam
voluptua. At vero eos et accusam et justo duo dolores et ea
rebum. Stet clita kasd gubergren, no sea takimata sanctus est Lorem
ipsum dolor sit amet.

\section{Background}

http://portal.acm.org/citation.cfm?id=1281670

http://wwwcg.in.tum.de/Research/data/Publications/eghw04.bib

http://portal.acm.org/citation.cfm?id=357320

\end{document}